\documentclass{scrbook}


\begin{document}

\title{Vyhodnoceni mereni rozmeru}

\subtitle{Pouzite pristroje}

\begin{center}
\begin{tabular}{ c c }
 \VAR{stroj1} & \VAR{chyba1} \\ 
 \VAR{stroj2} & \VAR{chyba2} \\  
 \VAR{stroj3} & \VAR{chyba3}    
\end{tabular}
\end{center}

\subtitle{Tabulky namerenych a zpracovanych hodnot}

\VAR{table1}

\VAR{table2}

Aritmeticky prumer:
[X = x = 1/n \sum_{k=1}^{n} X_k]
Pr.: [a = 1/10 \sum_{k=1}^{10} a_k = 10,48 mm]

Standardni nejistota typu A
[u_A(x) = \sqrt{1/n*(n-1)*\sum_{k=1}^{n} (X_k - X)^2}]
Pr.: [u_A(x) = \sqrt{1/10*(10-1)*\sum_{k=1}^{10} (a_k - a)^2} = 0,036 mm]

Standardni nejistota typu B
Osobni chyba (0,1 mm): [u_B1(X) = Z_1/k = 0,1/\sqrt{3} = 0,0577 mm]
Chyba meridla: mikrometr: [u_B21(X)= u_b(a) = u_b(d) = Z_21/k = 0,01/\sqrt{3} = 0,0058 mm]
			   posuvka: [u_B22(X)= u_b(b) = u_b(c) = u_b(v) = Z_22/k = 0,05/\sqrt{3} = 0,0289 mm]
			   
Kombinovana standardni nejistota
[u_c = \sqrt{u_A^2 + u_B^2}]

Vysledky jednotlivych parametru namerenych primou metodou:

a = \VAR{vys1}
b = \VAR{vys2}
c = \VAR{vys3}
d = \VAR{vys4}
v = \VAR{vys5}

\end{document}
